\documentclass[12pt]{article}
% \usepackage[left=2cm, right=2cm, top=1.5cm, bottom=1.5cm]{geometry}
\usepackage{amsmath}
\usepackage{amsthm}
\usepackage{amsfonts}
\usepackage{amssymb}
\usepackage{authblk}
\usepackage{tkz-euclide}
\usepackage{tikz}
\usepackage{changepage}
\usepackage{lipsum}
\usepackage{tree-dvips}
\usepackage{qtree}
\usepackage[linguistics]{forest}
\usepackage[hidelinks]{hyperref}
\usepackage{mathtools}
\usepackage{blindtext}
% \usepackage[cal=esstix,frak=euler,scr=boondox,bb= pazo]{mathalfa}
% the following 2 packages are used for changing the font. 
\usepackage{mathptmx}
\usepackage{mathrsfs}
\usepackage{graphicx}
\usepackage{setspace}
\graphicspath{{./images/}}
\allowdisplaybreaks
\allowbreak
\theoremstyle{definition}
\newtheorem{definition}{Definition}
\newtheoremstyle{named}{}{}{\itshape}{}{\bfseries}{.}{.5em}{\thmnote{#3's }#1}
\theoremstyle{named}
\newtheorem*{namedconjecture}{Distinct Factorizations Conjecture}
\newtheorem{conjecture}{Conjecture}
\DeclareMathOperator{\sech}{sech}
\DeclareMathOperator{\arcsec}{arcsec}
\DeclareMathOperator{\lcm}{lcm}
\DeclareMathOperator{\curl}{curl}
\DeclareMathOperator{\Res}{Res}
\DeclareMathOperator{\Aut}{Aut}
\DeclareMathOperator{\id}{id}
\newcounter{customDef}
\renewcommand{\thecustomDef}{\arabic{customDef}}
\newcommand{\Mod}[1]{\ (\mathrm{mod}\ #1)}
\begin{document}

\begin{center}
    Math 205 Theorems. 
\end{center}

\begin{enumerate}
    \item \textbf{Schwarz Lemma. } Let $f: \{z \in \mathbb{C}: |z| < 1\} \to \mathbb{C}$ be holomorphic and $|f(z)| \leq 1$ for all $z$, and $f(0)=0$. Then, $|f(z)| \leq |z|$ and $f'(0) \leq 1$. If for some $z_0 \neq 0$, $|f(z_0)| = |z_0|$ or if $|f'(0)| = 1$, then $f(z)=cz$ for some $c \in \mathbb{C}$ with $|c|=1$. 
    \item \textbf{Theorem. } Let $K \subseteq \mathbb{C}$ compact (write: $K \Subset \mathbb{C}$), $f: K \to \mathbb{C}$ continuous, $f$ holomorphic on $K$. Then, $\sup_{z \in K}|f(z)| = \sup_{z \in \partial K} |f(z)|$. 
    \item \textbf{Theorem. } Let $f: \Omega \to \mathbb{C}$ holomorphic ($\Omega$ open \& connected), $z_0 \in \Omega$, $|f(z_0)| = \sup_{z \in \Omega} |f(z)|$. Then, $f$ is constant. 
    \item \textbf{Theorem (Horwitz). } Let $\Omega \subseteq \mathbb{C}$ be open \& connected, $f: \Omega \to \mathbb{C}$, $f_n: \Omega \to \mathbb{C}$, $f_n$ holomorphic, $f_n(\Omega) \subset \mathbb{C} \setminus \{0\}$, $n \in \mathbb{N}$, $||f_n-f||_k \to 0$ for all $K \Subset \Omega$. Then, either $f=0$ identically or $f(\Omega) \subset \mathbb{C} \setminus \{0\}$. 
    \item \textbf{Theorem. } Let $\Omega \subseteq \mathbb{C}$ be open, $\mathscr{F}$ be a set of holomorphic function $\Omega \to \mathbb{C}$. Then, TFAE: 
    \begin{enumerate}
        \item For every $K \Subset \Omega$, $\sup_{f \in \mathscr{F}}||f||_K < \infty$. 
        \item For every sequence $(f_n)_{n \in \mathbb{N}} \subset \mathscr{F}$, there exists a subsequence $(f_{n_j})_{j \in \mathbb{N}}$, $n_1 < n_2 < \dots$, such that $(f_{n_j})_{j \in \mathbb{N}}$ is uniformly convergent on compact subsets of $\Omega$. 
    \end{enumerate}
    \item \textbf{Lemma. } Let $K \Subset \Omega$, $\mathscr{F}$ family of holomorphic functions $\Omega \to \mathbb{C}$ so that for every $K \Subset \Omega$, $\sup_{f \in \mathscr{F}} ||f||_K < \infty$. Given $\epsilon > 0$, there is a $\delta > 0$ such that $z,z'\in K$ and $|z-z'| < \delta$ imply $|f(z)-f(z')| < \epsilon$ for every $f \in \mathscr{F}$.
	\item \textbf{Riemman Mapping Theorem. } Let $\Omega \subset \mathbb{C}$ be open, connected, simply connected, and $\emptyset \neq \Omega \neq \mathbb{C}$. Then, $\Omega$ and $\mathbb{D} = \{|z| < 1\}$ are holomorphic and isomorphic (i.e. there exists a holomorphic $f: \Omega \to \mathbb{D}$ with holomorphic inverse). 
	\item \textbf{Prop. } Let $g \in SL_2(\mathbb{C})$. Then, $T_g \in \Aut(\mathbb{D})$ iff $g \in S \cap (1,1)$. 
	\item \textbf{Prop. } $\Aut\{\textrm{Im}z > 0\} = \{T_h \mid h \in SL_2(\mathbb{R}\}$.
	\item \textbf{Theorem. } Let $T_g$ be a fractional linear transformation and $z_1,z_2,z_3,z_4$ be distinct points in $\mathbb{C} \cup \{\infty\}$. Then, $(z_1,z_2,z_3,z_4) = (T_gz_1, T_g z_2, T_g z_3, T_g z_4)$. 
	\item \textbf{Lemma. } Let $g \in GL_2(\mathbb{C})$. Then, $\{w \in \mathbb{C} \cup \{\infty\} \mid T_{gw} \in \mathbb{R} \cup \{\infty\}\}$ is a circle on a (straight line) $\cup \{\infty\}$. 
\end{enumerate}  
 
\end{document}
