\input{~/normal-preamble.tex}

\begin{center}
    Math 205 Theorems. 
\end{center}

\begin{enumerate}
    \item \textbf{Schwarz Lemma. } Let $f: \{z \in \mathbb{C}: |z| < 1\} \to \mathbb{C}$ be holomorphic and $|f(z)| \leq 1$ for all $z$, and $f(0)=0$. Then, $|f(z)| \leq |z|$ and $f'(0) \leq 1$. If for some $z_0 \neq 0$, $|f(z_0)| = |z_0|$ or if $|f'(0)| = 1$, then $f(z)=cz$ for some $c \in \mathbb{C}$ with $|c|=1$. 
    \item \textbf{Theorem. } Let $K \subseteq \mathbb{C}$ compact (write: $K \Subset \mathbb{C}$), $f: K \to \mathbb{C}$ continuous, $f$ holomorphic on $K$. Then, $\sup_{z \in K}|f(z)| = \sup_{z \in \partial K} |f(z)|$. 
    \item \textbf{Theorem. } Let $f: \Omega \to \mathbb{C}$ holomorphic ($\Omega$ open \& connected), $z_0 \in \Omega$, $|f(z_0)| = \sup_{z \in \Omega} |f(z)|$. Then, $f$ is constant. 
    \item \textbf{Theorem (Horwitz). } Let $\Omega \subseteq \mathbb{C}$ be open \& connected, $f: \Omega \to \mathbb{C}$, $f_n: \Omega \to \mathbb{C}$, $f_n$ holomorphic, $f_n(\Omega) \subset \mathbb{C} \setminus \{0\}$, $n \in \mathbb{N}$, $||f_n-f||_k \to 0$ for all $K \Subset \Omega$. Then, either $f=0$ identically or $f(\Omega) \subset \mathbb{C} \setminus \{0\}$. 
    \item \textbf{Theorem. } Let $\Omega \subseteq \mathbb{C}$ be open, $\mathscr{F}$ be a set of holomorphic function $\Omega \to \mathbb{C}$. Then, TFAE: 
    \begin{enumerate}
        \item For every $K \Subset \Omega$, $\sup_{f \in \mathscr{F}}||f||_K < \infty$. 
        \item For every sequence $(f_n)_{n \in \mathbb{N}} \subset \mathscr{F}$, there exists a subsequence $(f_{n_j})_{j \in \mathbb{N}}$, $n_1 < n_2 < \dots$, such that $(f_{n_j})_{j \in \mathbb{N}}$ is uniformly convergent on compact subsets of $\Omega$. 
    \end{enumerate}
    \item \textbf{Lemma. } Let $K \Subset \Omega$, $\mathscr{F}$ family of holomorphic functions $\Omega \to \mathbb{C}$ so that for every $K \Subset \Omega$, $\sup_{f \in \mathscr{F}} ||f||_K < \infty$. Given $\epsilon > 0$, there is a $\delta > 0$ such that $z,z'\in K$ and $|z-z'| < \delta$ imply $|f(z)-f(z')| < \epsilon$ for every $f \in \mathscr{F}$.
	\item \textbf{Riemman Mapping Theorem. } Let $\Omega \subset \mathbb{C}$ be open, connected, simply connected, and $\emptyset \neq \Omega \neq \mathbb{C}$. Then, $\Omega$ and $\mathbb{D} = \{|z| < 1\}$ are holomorphic and isomorphic (i.e. there exists a holomorphic $f: \Omega \to \mathbb{D}$ with holomorphic inverse). 
	\item \textbf{Prop. } Let $g \in SL_2(\mathbb{C})$. Then, $T_g \in \Aut(\mathbb{D})$ iff $g \in S \cap (1,1)$. 
	\item \textbf{Prop. } $\Aut\{\textrm{Im}z > 0\} = \{T_h \mid h \in SL_2(\mathbb{R}\}$.
	\item \textbf{Theorem. } Let $T_g$ be a fractional linear transformation and $z_1,z_2,z_3,z_4$ be distinct points in $\mathbb{C} \cup \{\infty\}$. Then, $(z_1,z_2,z_3,z_4) = (T_gz_1, T_g z_2, T_g z_3, T_g z_4)$. 
	\item \textbf{Lemma. } Let $g \in GL_2(\mathbb{C})$. Then, $\{w \in \mathbb{C} \cup \{\infty\} \mid T_{gw} \in \mathbb{R} \cup \{\infty\}\}$ is a circle on a (straight line) $\cup \{\infty\}$.
	\item \textbf{Theorem. } Let $\Omega$ be an open, connected set so that there is $f: \Omega \to \mathbb{D}$ that is a holomorphic isomorphism. Then, $\textrm{iso}(\Omega, \mathbb{D}) \ni g \to \left(g(z_0),\frac{g'(z_0)}{|g'(z_0)|}\right) \in \mathbb{D} \times \{|z|=1\}$ is a bijection. 
	\item \textbf{Definition of Jordan Curve. } A Jordan Curve is given by a map $[0,1] \ni t \to C(t) \in \mathbb{C}$ which is continuous, 1-1 on $[0,1)$ and $C(0)=C(1)$ (no self-intersection otherwise). 
	\item \textbf{Jordan Curve Theorem. } If $C: [0,1] \to \mathbb{C}$ is a Jordan curve, then $\mathbb{C} \setminus C([0,1])$ has 2 connected components. One of these is bounded and the other, unbounded. (The bounded component is called the "interior") We shall denote by $|C|$ the set $C([0,1])$ when $C: [0,1] \to \mathbb{C}$. 
	\item \textbf{Caratheodory's Theorem. } Let $\Gamma$ be a Jordan curve and $\Omega$ be the interior region (then $\partial \Omega = |\Gamma|$). Then, if $f: \mathbb{D} \to \Omega$ is a holomorphic isomorphism, then $f$ extends to a homeomorphism $\overline{\mathbb{D}} \to \overline{\Omega}$, where $\partial \mathbb{D}$ is mapped to $\partial \Omega = |\Gamma|$.
	\item  \textbf{Rectifiable def. } An arc $\phi: [a,b] \to \mathbb{C}$ (the map $\phi$ is 1-1 and continuous) is rectifiable if it has 'length' (bounded variation), that is, if: 
	$$
	\sup_{a=t_0<t_1<\dots<t_k=b} \sum_{j=0}^{k-1} \left(|\phi(t_{j+1}) - \phi(t_j)|\right) < \infty
	$$
	where $k \in \mathbb{N}$. 
	\item \textbf{Theorem. } Let $\Omega,\omega$ be disjoint open regions and $\Gamma$ a rectifiable arc so that $|\Gamma| \subset \partial \Omega \cap \partial \omega$ and $|\Gamma| \cup \Omega \cup \omega$ open ($\Gamma$ has no endpoints). Assume also $f: |\Gamma| \cup \Omega \to \mathbb{C}$, $g: |\Gamma| \cup \omega \to \mathbb{C}$ are continuous and $f\mid_\Omega$, $g\mid_\Omega$ holomorphic and $f\mid_{|\Gamma|} = g\mid_{|\Gamma|}$. Then, $F: \Omega \cup |\Gamma| \cup \omega \to \mathbb{C}$ is holomorphic.
	\item \textbf{Theorem (Schwarz Reflection Principle). } Let $\Omega = \Omega^{\star}$ ($=\{\overline{z} \mid z \in \Omega\}$) open region, $\Omega \cap \mathbb{R} \supset (a,b)$ and $\Omega_{\pm} = \Omega \cap \{\pm \textrm{Im}z > 0\}$. If $f: \Omega_+ \cup (a,b) \to \mathbb{C}$ continuous, $f\mid_{(a,b)} \subset \mathbb{R}$, $f\mid_{\Omega_+}$ holomorphic, then, $F(z)$, with $F(z)=f(z)$ if $z \in \Omega_+ \cup (a,b)$ and $F(z)=\overline{f(\overline{z})}$ if $z \in \Omega_-$ is holomorphic in $\Omega_+ \cup (a,b) \cup \Omega_-$. 
	\item \textbf{Analytic Arc def. } Analytic arc is $\phi: (a,b) \to \mathbb{C}$ so that there is $f: \omega \to \mathbb{C}$ univalent with $\omega \supset (a,b), f \mid_{(a,b)} = \phi$, where we also require $\phi$ to be holomorphic within some neighborhood containing it. 
	\item \textbf{Theorem. } Let $\Omega$ be a region, $\gamma$ an analytic arc, $|\gamma| \supset \partial \Omega$ from univalent $f: \omega \to \mathbb{C}$ and assume the following: 
	\begin{enumerate}	
		\item $f(\omega \cap \{\textrm{Im}z > 0\}) \subset \Omega$. 
		\item $f(\omega \cap \{\textrm{Im}z < 0\}) \cap \Omega = \emptyset$. 
		\item let $F: \Omega \cup |\gamma| \to \mathbb{C}$ continuous, and $F \mid_\Omega$ holomorphic with $F(|\gamma|) \subset |\Gamma|$, where $\Gamma$ is an analytic arc. 
	\end{enumerate} 
	Then, there is an open $\Omega_1$, with $\Omega_1 \supset \Omega \cup |\gamma|$ so that $F$ has a holomorphic extension to $\Omega_1$. 
	\item \textbf{Theorem (Schwarz-Christoffel Formula). } Let $F: \overline{\mathbb{D}} \to \overline{\Omega}$ be a homeomorphism (by Caratheodory) which extends the conformal map $F \mid_{\mathbb{D}} \to \Omega$ and $F(w_k) = z_k$. Let $\overline{\Omega}$ be a polygon with angles $\alpha_k\pi, \beta_k = 1-\alpha_k$. Then, 
	$$
	F(w) = C \cdot \left(\int_{0}^{w} \left(\prod_{k=1}^{n} (w-w_k)^{-\beta_k}\right) dw \right) + C'. 
	$$
	\item \textbf{Theorem. } If $\gamma$ is an analytic arc, then it is automatically rectifiable. 
	\item \textbf{Schwarz-Christoffel Formula for Upper-Half Plane. } If $G: \{\textrm{Im} u > 0\} \to \Omega$ is a conformal map, where $\Omega$ is the interior of a polygon with outer angles $\beta_1\pi,\dots,\beta_k\pi$ and the point $\infty$ corresponds to $z_n$, then: 
	$$
	G(u) = C \cdot \left(\int_{0}^{u} \left(\prod_{k=1}^{n-1} (u - \xi_k)^{-\beta_k}\right) du \right) + C', 
	$$
	where $\xi_k \in \mathbb{R}$. The product has only $n-1$ factors. The external angle $\beta_n$ does not appear explicitly. If $\beta_1 + \dots + \beta_{n-1}=2$, then $\beta_n = 0$. 
	\item \textbf{Schwarzian Derivative. } For a function $f$, the Schwarzian derivative of $f$ is defined as: 
	$$
	S(f) = \frac{f'''}{f'} - \frac{3}{2}\left(\frac{f''}{f'}\right)^2 = \left(\frac{f''}{f'}\right)' - \frac{1}{2}\left(\frac{f''}{f'}\right)^2. 
	$$
	\item \textbf{A formula using Schwarzian derivative. } $S(f \circ g) = (S(f) \circ g)(g')^2 + S(g)$. 
	\item \textbf{Cor. } Now let $f(z) = \frac{az+b}{cz+d}$ be a fractional linear transformation. Then, $S(f)=0$. Also, we get that $S(f \circ g) = S(g)$. Thus, we conclude that $S(g)$ is invariant under composition with a fractional linear transformation, under the Schwarzian derivative operator. 
	\item \textbf{$\Gamma$ Free Group def. } This is defined to be $\Gamma := \left\langle \begin{pmatrix} 1 & 2 \\ 0 & 1\end{pmatrix}, \begin{pmatrix} 1 & 0 \\ 2 & 1 \end{pmatrix}\right\rangle$, with the two listed matrices as its generators. Also, we have that $\Gamma$ is a subgroup of $SL_2(\mathbb{Z})$. 
	\item \textbf{Prop. } Let $y_1,y_2$ be two linearly independent solutions to $y'' + py=0$. Then, $u = \frac{y_1}{y_2}$ is so that $S(u)=2p$, where $S$ is the Schwarzian derivative operator. 
	\item \textbf{Modular Function def. } Consider the free group $\Gamma$ as defined two items above. Now, consider $\Gamma$ except now, $b \equiv c \equiv 0 \Mod{2}$. Define a function $\lambda: S \to \mathbb{H}$, where $\lambda$ takes $0,1,\infty$ to $1,\infty,0$, respectively (here, $S$ refers to domain from class based on the conformal mapping operated on the sides of the non-Euclidean triangle; namely, $S = \{0 < \textrm{Re} z < 1\} \setminus \{\frac{1}{2} + z \mid |z| < \frac{1}{2}\}$). 
	\item \textbf{Picard's Theorem. } Let $g: \mathbb{C} \to \mathbb{C}$ be entire. If there exists at least two points in $\mathbb{C} \setminus \textrm{range}(g)$, tehn $g$ is constant. 
	\item \textbf{Mittag-Leffler Theorem. } Given $b_n \in \mathbb{C}$, $n \in \mathbb{N}$, $\lim_{n \to \infty} |b_n| = \infty$ and principal parts $P_n = \sum_{k=-N_m}^{-1} c_k^{(n)}(z-b_n)^k$ with $c_{-N_m} \neq 0$. Then, there is a meromorphic function on $\mathbb{C}$ with poles $(b_n)_{n \in \mathbb{N}}$ and principal parts $P_n$ of the Laurent expansions at the poles. 
	\item \textbf{Formula. } $\frac{\pi^2}{(\sin(\pi z))^2} = \sum_{n \in \mathbb{Z}} \frac{1}{(z-n)^2}$. 
	\item \textbf{Formula. } $\lim_{N \to +\infty} \sum_{|n| \leq N} \frac{1}{z-n} = \pi \cdot \cot(\pi z)$. 
	\item \textbf{Infinite product convergence def. } $\prod_{k \geq 1} z_k$ converges iff $\lim_{k \to \infty} \prod_{i=1}^{k}z_i$ exists and is nonzero. 
	\item \textbf{Notation. } Let $\log z := \{a \in \mathbb{C} \mid e^a = z\}$. Let $\Log z := a + i(-\pi,\pi]$, with $a \in \mathbb{R}$. Similarly, let $\arg z := \textrm{Im} \log z$ and let $\Arg z := \textrm{Im} \Log z$. 
	\item \textbf{Theorem. } $\prod_{k \geq 1}z_k$ converges iff $\sum_{k \geq 1} \Log z_k$ converges. 
	\item \textbf{Theorem. } $\prod_{k \geq 1} z_k$ converges implies $z_k \to 1$. 
	\item \textbf{Infinite proudct absolute convergence def. } $\prod_{k \geq 1}z_k$ is absolutely convergent iff $\sum_{k \geq 1} |\Log z_k| < \infty$. 
	\item \textbf{Theorem. } $\prod_{k \geq 1}z_k$ is absolutely convergent iff $\sum_{k \geq 1} |z_k - 1| < \infty$. 
	\item \textbf{Weierstrass Theorem. } Given $a_n \in \mathbb{C}$, $|a_n| \to \infty$, $a_n \neq 0$, and $n \geq 0$ an integer, there exists an entire function with multiplicity of 0 at zero 0 and other zeros at $a_n$ (multiplicities by repetition). Every function with these zeros is of the form 
	$$
	f(z) = z^m \cdot e^{g(z)} \cdot \prod_{n = 1}^{\infty} \left(1 - \frac{z}{a_n}\right) \cdot e^{\frac{z}{a_n} + \dots + \frac{z^{k_n}}{a_n^{k_n}} \cdot \frac{1}{k_n}}
	$$
	for some $k_n \geq 0$, $g$ entire, and the infinite product uniformly absolutely convergent on compact subsets of $\mathbb{C}$. 
	\item \textbf{Cor. } If $f$ is meromorphic on $\mathbb{C}$, then there are $f_1,f_2$ entire functions so that $f = \frac{f_1}{f_2}$. 
	\item \textbf{Canonical product, genus def. } If we have that the sum in the exponential of $e$ in the infinite product component of $f$ (as in Weierstrass theorem) has last term that is raised to a fixed exponent $h$, we say $f$ is the canonical product and say that $f$ has genus $h$. Equivalently, we say if $f: \mathbb{C} \to \mathbb{C}$ entire, we say $f$ has finite genus if $f = e^{g(z)} \cdot P(z)$, where $P(z)$ is a canonical product and $g$ is a polynomial. 
	\item \textbf{Order of growth def. } Let $f: \mathbb{C} \to \mathbb{C}$ be holomorphic. Then the order of growth of $f$ is 
	$$
	\rho = \limsup_{R \to \infty} \frac{\log(\log ||f||_{R \mathbb{D}})}{\log R} = \inf \{m \geq 0 \mid |f(z)| \leq Ce^{c|z|^m}\}. 
	$$
	\item \textbf{Hadamard's Theorem. } If $\rho$ and $h$ are the order of growth and the genus of an entire function of finite genus respectively, then $h leq \rho \leq h+1$. 
	\item \textbf{Cor. } If $\rho$ is fractional, the entire function takes every value infinitely many times. 
	\item \textbf{$\gamma$, Euler-Mascheroni Constant. } $\gamma = \lim_{N \to \infty} \left(-\log N + (1 + \dots + \frac{1}{N})\right)$. 
	\item \textbf{Formula. } $\frac{\pi}{\sin\pi z} = \Gamma(z)\Gamma(1-z)$. 
	\item \textbf{Equivalent definition of $\Gamma$-function. } $\Gamma(z) = \int_{0}^{\infty}t^{z-1}e^{-t} dt$. 
	\item \textbf{Stirling's Formula. } $\Gamma(z) = \sqrt{2\pi} \cdot z^{z - \frac{1}{2}} \cdot e^{-z} \cdot e^{\mathscr{J}(z)}$ ($\textrm{Re}z>0$), where $\mathscr{J}(z) = \frac{1}{\pi} \cdot \int_{0}^{\infty} \frac{z}{\eta^2 + z^2} \cdot \log \left(\frac{1}{1 - e^{-2\pi\eta}}\right) d\eta$. 
	\item \textbf{Fourier transform def. } The Fourier transform of $f$ (on the real line) at $x \in \mathbb{R}$ is $\mathscr{F} f(x) = \int_{\mathbb{R}} f(t) \cdot e^{ixt} dt$. 
	\item \textbf{Mellin transform def. } This is a Fourier transform on $((0,\infty), \cdot)$ $\int_{0}^{\infty} \lambda^z \cdot f(\lambda) \cdot \frac{d \lambda}{\lambda} = \int_{0}^{\infty} \lambda^{z-1} \cdot f(\lambda) \cdot d\lambda$. To get $I(z)$, one takes $f(\lambda) = e^{-\lambda}$, for $\lambda \in (0,\infty)$. 
	\item \textbf{Lemma. } $\int_{0}^{\infty} t^z e^{-\lambda t} \frac{dt}{t} = \lambda^{-z} \Gamma(z)$, for $\lambda>0$ and $\textrm{Re}z>0$. 
	\item \textbf{Formula. } Take $g(t) = \sum_n c_n e^{\lambda_n t}$ with $\lambda_n \to \infty$, with $\lambda_n > 0$. Then, the Mellin transform of $g(t)$ is $\int_{0}^{\infty} t^z (\sum_n c_n e^{-\lambda_n t}) \frac{dt}{t} = \Gamma(z) \cdot \sum_n c_n \lambda_n^{-z}$. 
	\item \textbf{Formula. } $\frac{1}{\Gamma(z)} = \lim_{n \to \infty} \frac{1}{n!} \left(1 - \frac{\log n}{n}z\right)^n \cdot \prod_{m=0}^n (z+m)$. 
	\item \textbf{Formula. } $\theta(t) = \frac{1}{\sqrt{t}} \cdot \theta(\frac{1}{t})$ if $t>0$, where $\theta(t) = \sum_{n \in \mathbb{Z}} e^{-\pi n^2 t}$. 
	\item \textbf{Formula. } The Poisson Summation Formula is roughly that for \enquote{good $f$}, $\sum_{n \in \mathbb{Z}} f(n) = \sum_{n \in \mathbb{Z}} \mathscr{F}f(n)$. 
	\item \textbf{Prop. } If $f: \mathbb{R} \to \mathbb{C}$ so that $f$ is continuous $\sum_{n \in \mathbb{Z}} ||f^{(k)}||_{[n,n+1]} < \infty$ for $k=0,1,2,\dots$, then the Poisson Summation Formula $\sum_{n \in \mathbb{Z}} f(n) = \sum_{n \in \mathbb{Z}} \mathscr{F}f(n)$ holds. 
	\item \textbf{Cor. } If $f: \mathbb{R} \to \mathbb{C}$, $C^2$ and $|f|(\lambda) \leq C(1+t^2)^{-1}$, $|f'|(t) \leq C(1+t^2)^{-1}$, $|f''(t)| \leq C(1+t^2)^{-2}$, then the Poisson Summation formula holds for $f$. 
	\item \textbf{Cor. } Let $\lambda>0$. Then the Poisson Summation Formula holds for $f(t) = e^{-\lambda t^2}$. 
	\item \textbf{Lemma. } $\int_{-\infty}^{\infty} e^{-\pi t^2} dt = 1$, which is the Fourier transform of $e^{-\lambda t^2}$. 
	\item \textbf{Prop. } If $\textrm{Re} z > \frac{1}{2}$, then $\xi(2z)\Gamma(z) \cdot \pi^{-z} = \int_{1}^{\infty} \left(t^{z-1}+ t^{-z - \frac{1}{2}}\right) \cdot \psi(t) dt + \frac{1}{2z(2z-1)}$. 
	\item \textbf{Prop. } $\pi^{-\frac{w}{2}} \cdot \xi(w) \cdot \Gamma(\frac{w}{w}) = \pi^{-\frac{1-w}{2}} \cdot \xi(1-w) \cdot \Gamma(\frac{1-w}{2})$. 
	\item \textbf{Euler Product Formula. } $\zeta(z) = \prod_{n \geq 1} (1 - p_n^{-z})^{-1}$, where $p_1,p_2,\dots$ is the sequence of primes. 
	\item \textbf{Euler-Beta Function. } This is the function $\beta(z,w) = \int_{0}^{1} (1-t)^{z} \cdot t^{w-1} dt$. 
	\item \textbf{Characteristic Jacobi $\Theta$-Function def. } $\Theta_{a,b}(z,\tau) = \sum_{n \in \mathbb{Z}} e^{\pi i(n+a)^2 \tau + 2\pi i(n+a)(z+b)}$. 
	\item \textbf{Special operators. } We define $S_b$ and $T_a$ to be linear operators on the space $\mathscr{F}(\mathbb{C} \to \mathbb{C})$ to be $(S_b f)(z) = f(z+b)$ and $(T_a f)(z) = f(z+a\tau) \cdot e^{\pi a^2 \tau + 2\pi i az}$. 
	\item \textbf{Lemma. } $\Theta(z_0,\tau) = 0$ iff $z_0 \in (\mathbb{Z} + \frac{1}{2})\tau + \mathbb{Z}$ and are simple zeros. 
	\item \textbf{Cor. } $\Theta_{a,b}(z,\tau)$ has simple zeros which are located at $\frac{-2a+1}{2}\tau + \frac{-2b+1}{2} + (\mathbb{Z} + \tau\mathbb{Z})$. 
	\item \textbf{Formula. } $S_b T_a = e^{2\pi iab} T_a S_b$. 
	\item \textbf{Heisenberg group. } This is the group $N = \{
	\begin{pmatrix}
	1 & x & z \\
	0 & 1 & y \\
	0 & 0 & 1
	\end{pmatrix}
	\mid (x,y,z) \in \mathbb{R}^3\}$, with matrix multiplication as the group operation. We also define the subgroup $\Gamma \subseteq N$ to be $\Gamma = \{
	\begin{pmatrix}
	1 & x & z \\
	0 & 1 & y \\
	0 & 0 & 1
	\end{pmatrix}
	\mid (x,y,z) \in \mathbb{Z}^3 \}$. We write elements of the Heisenberg group as $[x,y,z]$, for brevity. 
	\item \textbf{Representation of Heisenberg Group. } Put $\rho([a,b,c]): \mathscr{F}(\mathbb{C}) \to \mathscr{F}(\mathbb{C})$ by $\rho([a,b,c])(f) = e^{2\pi i c} T_b S_a f$. 
\end{enumerate}  
 
\end{document}
