\input{~/normal-preamble.tex}

\begin{center}
	Math 205 - HW 4
\end{center} 

\noindent \textit{Solution: } Vignesh Nydhruva.  
\\ 
\\
\textit{Problem: } Let $\lambda = (z_1,z_2,z_3,z_4) = \frac{z_1 - z_3}{z_1 - z_4} / \frac{z_2 - z_3}{z_2 - z_4}$ (the cross ratio) and let $\sigma$ be a permutation on $\{1,2,3,4\}$. Then, prove that the value of $\sigma(\lambda) := (z_{\sigma(1)}, z_{\sigma(2)}, z_{\sigma(3)}, z_{\sigma(4)})$ is one of $\lambda, \frac{1}{\lambda}, 1 - \lambda, \frac{1}{1-\lambda}, 1 - \frac{1}{\lambda}, \frac{\lambda}{\lambda - 1}$. 
\\
\\
\textit{Proof: } The main idea of this proof is to introduce the operation of swapping elements in a cross ratio and observe its connection to functions of $\lambda$, along with their compositions. For brevity, the algebra calculations have been omitted. First, note that swapping $z_1$ and $z_2$ in the cross ratio for $\lambda$ gives $(z_2,z_1,z_3,z_4) = \frac{1}{\lambda}$. Swapping $z_3$ and $z_4$ instead gives $(z_1,z_2,z_4,z_3) = \frac{1}{\lambda}$. Since both of these swaps give the same function of $\lambda$, we declare them equivalent; we thus have that $z_1 \leftrightarrow z_2$ or $z_3 \leftrightarrow z_4$ gives $\lambda \xmapsto{S_1} \frac{1}{\lambda}$. We also consider the other types of swaps that can be made: $z_1 \leftrightarrow z_3$ or $z_2 \leftrightarrow z_4$ gives $\lambda \xmapsto{S_2} \frac{\lambda}{\lambda - 1}$ and lastly, $z_1 \leftrightarrow z_4$ or $z_2 \leftrightarrow z_3$ gives $\lambda \xmapsto{S_3} 1 - \lambda$. It follows trivially that any permutation of the elements in the cross ratio $(z_1,z_2,z_3,z_4)$ can be formed by composing the swap operations previously listed. It can also be verified that if $\sigma$ preserves the positions of all except two of the $z_i$'s then $\sigma(\lambda)$ is one of $\frac{1}{\lambda}, \frac{\lambda}{\lambda - 1}, 1 - \lambda$. Also, recognize that if a permutation consists of two nonequivalent swaps, then none of $z_i$'s are in the original placement in $\sigma(\lambda)$; this can be proved by exhaustion by considering the possibilities of nonequivalent swap pairs and their composition. By the definition of a swap (that acts on two elements), it follows that the number of elements that retain their position after $\sigma(\lambda)$ is even - in this case it is either 4, 2, or 0. Furthermore, it can be verified (by exhaustion) that any permutation $\sigma$ on $\{1,2,3,4\}$ can be obtained by performing at most 3 swaps. If we perform 0 swaps, we have the identity permutation thus giving the identity map, $(z_1,z_2,z_3,z_4) \xmapsto{\sigma} (z_1,z_2,z_3,z_4) = \lambda$, so $\lambda \mapsto \lambda$. If we perform 1 swap, then, as mentioned previously, $\sigma(\lambda)$ is one of $\lambda, 1-\lambda, \frac{\lambda}{\lambda - 1}$. If we perform 2 nonequivalent swaps, then $\sigma(\lambda)$ preserves the position of none of the $z_i$'s. It then follows that by the choice of the maps $S_1,S_2,S_3$ (each given by 2 equivalent swaps), we construct the bijection $A:= \{S_1,S_2,S_3\} \longleftrightarrow B:= \{\{z_1 \leftrightarrow z_2, z_3 \leftrightarrow z_4\}, \{z_1 \leftrightarrow z_3, z_2 \leftrightarrow z_4\}, \{z_1 \leftrightarrow z_4, z_2 \leftrightarrow z_3\}\}$, with $a_i \longleftrightarrow b_i$ for $i \in \{1,2,3\}, a_i \in A, b_i \in B$. Thus, considering the swaps in any $b_i$ as equivalent, we have that any map $S_i$ acting on $\lambda$ corresponds to a particular swap, and vice-versa. Thus, composing distinct (respectively, indistinct) $S_i$ maps is equivalent to composing nonequivalent (respectively, equivalent) swaps. Now consider the possible compositions of two nonequivalent $S_i$ maps. Permuting the indices in $S_i \circ S_j$ (letting $i,j \in \{1,2,3\}$ with $i \neq j$) gives the maps $\lambda \mapsto \frac{1}{1 - \lambda}, \lambda \mapsto 1 - \frac{1}{\lambda}$. However, the final case to consider is when $\sigma$ can only be formed from performing 3 swaps, with nonconsecutive repetition allowed. Then, all possible permutations $\sigma$ are considered and hence, it can be verified that each of these 3-map compositions gives a value in the list $\lambda, \frac{1}{\lambda}, 1 - \lambda, \frac{1}{1 - \lambda}, 1 - \frac{1}{\lambda}, \frac{\lambda}{\lambda - 1}$. $\mathbf{Q.E.D.}$ 

\end{document}
